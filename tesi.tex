\documentclass[12pt,a4paper,openright,twoside]{report}
\usepackage[english]{babel}
\usepackage{fancyhdr}
\usepackage{indentfirst}
\usepackage{newlfont}
\usepackage{pdfpages}
\usepackage{colortbl}
\usepackage{afterpage}
\usepackage{float}
\usepackage{graphicx}
\usepackage{listings}
\usepackage{subcaption}
\usepackage[font={small,it}]{caption}
\usepackage{emptypage}
\usepackage[hidelinks]{hyperref}
\usepackage{hhline}
\usepackage{amsmath}
\usepackage{amssymb}
\usepackage{amsfonts}
\usepackage{mathtools}
\usepackage{multirow}
\usepackage{longtable}
\usepackage{url}

\urlstyle{tt}  % or \urlstyle{tt} for typewriter font
\def\UrlBreaks{\do\/\do-\do_\do.\do?\do&\do=}


\renewcommand{\lstlistingname}{Codice}
\definecolor{background}{HTML}{EEEEEE}
\lstdefinelanguage{nginx}{
	basicstyle=\footnotesize\ttfamily,
    numbers=left,
    numberstyle=\ttfamily,
    stepnumber=1,
    numbersep=8pt,
    showstringspaces=false,
    breaklines=true,
	belowskip=2em,
	aboveskip=2em,
    backgroundcolor=\color{background}
	}
	
\graphicspath{ {./images/} }
\oddsidemargin=30pt \evensidemargin=20pt
\pagestyle{fancy}\addtolength{\headwidth}{20pt}\setlength{\headheight}{15pt}
\renewcommand{\chaptermark}[1]{\markboth{\thechapter.\ #1}{}}
\renewcommand{\sectionmark}[1]{\markright{\thesection \ #1}{}}
\rhead[\fancyplain{}{\bfseries\leftmark}]{\fancyplain{}{\bfseries\thepage}}
\cfoot{}
\linespread{1.3}

\begin{document}
\includepdf[pages=-]{titlepage.pdf}
\clearpage{\pagestyle{empty}\cleardoublepage}
\begin{titlepage}
	\setcounter{page}{3}
	\thispagestyle{empty}
	\topmargin=6.5cm
	\raggedleft
	\large
	\em
	Alla migliore madre del mondo,\linebreak
	al miglior padre del mondo,\linebreak
	e al miglior fratello del mondo.
	\newpage
	\clearpage{\pagestyle{empty}\cleardoublepage}
\end{titlepage}
\clearpage{\pagestyle{empty}\cleardoublepage}
\selectlanguage{english}
\begin{abstract}
	\setcounter{page}{3}
	\pagenumbering{roman}
{	\small
	Large Language Models have transformed our interaction with technology, yet their massive scale introduces major drawbacks. Their substantial energy consumption and water usage create environmental pressures, while cloud-based architectures limit user autonomy and enable pervasive data collection. Additionally, their resource-intensive nature leads to entirely different challenges for edge deployment, where memory constraints and computational limitations make standard LLMs impractical.

	This thesis presents a comprehensive compression pipeline that systematically combines multiple optimization strategies, beginning with the compact LLaMA 3.2 1B Instruct as the foundation model: depth pruning removes redundant transformer layers based on importance metrics, width pruning applies structured sparsity to attention matrices using the WANDA algorithm, Low-Rank Adaptation recovers performance while enabling task-specific fine-tuning, 4-bit quantization further reduces memory footprint through GPTQ, and Eigenspace Low-Rank Approximation provides training-free accuracy recovery.

	The experimental evaluation demonstrates that integrating multiple optimization techniques can significantly decrease memory footprint while preserving core language functionality. Recovery procedures, particularly LoRA adaptation, prove highly effective at restoring capabilities lost during pruning and mitigating performance degradation. However, applying these methods beyond specific thresholds leads to rapid deterioration that even recovery procedures cannot fully address, indicating fundamental architectural limits to transformer compression.
	}
\end{abstract}
\clearpage{\pagestyle{empty}\cleardoublepage}
\selectlanguage{english}
\clearpage{\pagestyle{empty}\cleardoublepage}
\tableofcontents
\addtocontents{toc}{\protect\thispagestyle{empty}}
\listoffigures
\clearpage{\pagestyle{empty}\cleardoublepage}
\listoftables
\clearpage{\pagestyle{empty}\cleardoublepage}
\pagenumbering{arabic}

\chapter{Introduction}
\label{chap:intro}
\lhead[\fancyplain{}{\bfseries\thepage}]{\fancyplain{}{\bfseries\rightmark}}
It is no secret that, in the last three years, \textit{Large Language Models} (LLMs) have fundamentally transformed our relationship with technology.
Their impact rivals the most significant innovations of the past century, such as the internet and the smartphone. When people contemplate Artificial Intelligence today, they immediately think of ChatGPT or Claude, which have seamlessly integrated into our daily routines.
Yet these powerful tools come with significant environmental concerns. Their development and operation consume vast amounts of energy and water resource: modern data centers supporting these models require extensive cooling systems and electricity consumption that can rival small cities.

The computational complexity of these systems necessitates cloud-based deployment, which not only amplifies their environmental footprint but also fundamentally restricts user autonomy. This cloud dependency creates a concerning power dynamic where users have limited control over their tools, while simultaneously enabling extensive data collection practices and potential surveillance mechanisms that would be impossible with local, user-controlled alternatives.

In this more conversational opening chapter we will briefly examine the environmental and social impact of LLMs while highlighting the growing imperative for efficient, locally-deployable models that democratize access without depleting our planet's resources. Afterwards, we will outline the scope of this project, which aims to explore the potential of compression techniques to make LLMs more efficient.

The future of AI depends not just on what these models can do, but how sustainably they can do it.

\section{The social impact of LLMs}
The widespread adoption of LLMs has created ripple effects across virtually every sector of society, fundamentally altering how we work, learn, and create. In education, these tools have sparked heated debates about academic integrity while simultaneously offering new possibilities for personalized learning and accessibility for students with disabilities \cite{academic_integrity}. The workplace has experienced perhaps the most dramatic shifts, with entire professions grappling with automation anxiety while others discover unprecedented productivity gains. Creative industries find themselves in particularly complex territory: writers, artists, and content creators must navigate between leveraging AI as a collaborative tool and protecting their intellectual property from being absorbed into training datasets without consent or compensation \cite{artistic_integrity}.

What strikes me most profoundly is how these models have democratized access to sophisticated capabilities that were once the exclusive domain of experts. A small business owner can now generate marketing copy that rivals professional agencies, students can receive tutoring in subjects where human expertise might be scarce, and non-programmers can write functional code with natural language instructions. Yet this democratization comes with a troubling caveat: it's entirely dependent on maintaining access to centralized, corporate-controlled systems. When OpenAI experiences an outage, millions of users worldwide suddenly lose access to tools they've integrated into their daily workflows. When pricing models change, entire business models built around AI assistance can become unsustainable overnight.

This dependency becomes even more concerning when we consider the data these systems collect. Every interaction, every query, every creative prompt may potentially become part of these companies' datasets, raising questions about privacy and intellectual property.
As such, the need for transparency and user control over these systems has never been more urgent, and I personally believe that the future of AI must prioritize local, user-deployable models that empower individuals rather than centralizing power in the hands of a few corporations.

\section{The environmental impact of LLMs}

\section{Scope of this project}

\section{Document structure}
Pending for later.

\chapter{Background and Related Work}
\label{chap:background}
\lhead[\fancyplain{}{\bfseries\thepage}]{\fancyplain{}{\bfseries\rightmark}}
% Chapter 2: Background and Related Work
Before explaining the details and implementation of the methodology used in this project, it is essential to provide an overview of the evolution of the inner workings of the Transformer architecture as well as Large Language Models. In addition, we will also discuss revelant compression techniques that have been developed in this context, and how they influenced this work.
Finally, we will also shed some light on the target hardware, whose limitations have been a driving force behind the design choices made in this project.

\section{The architecture of Transformers}

% [explain SLMs here]
\section{The structure of Large Language Models}

\section{Relevant compression techniques}




\chapter{Methodology and Implementation}
\label{chap:methodology}
\lhead[\fancyplain{}{\bfseries\thepage}]{\fancyplain{}{\bfseries\rightmark}}
Having covered the importance of LLM optimization and the technical architecture of transformers in previous chapters, we now turn to the practical challenge of model compression.

This chapter begins with preliminary experiments on layer manipulation that revealed crucial insights about architectural redundancy in transformer models and shaped the main compression strategy. The chapter then presents what is possibly the heart of this research: a multi-stage pipeline that combines several optimization techniques in a structured sequence. Each stage is examined with both theoretical foundations and practical implementation details. Finally, the evaluation methods and their underlying motivation are detailed.

\section{Preliminary Research: Franken-LLaMA} \label{frankenllama}

Before embarking on systematic compression techniques for our target 1B parameter LLaMA model, we conducted preliminary research to understand the behavior of transformer architectures under structural modifications. This exploratory consisted on the ``Franken-LLaMA'' project \cite{franken-llama}, which involved experimenting with selective layer skipping and repetition in the larger LLaMA2-7B-Chat \cite{llama2} model to gain a first insight into which components of the transformer architecture are most critical for maintaining model performance.

The approach centered on modifying the standard transformer execution flow by selectively including, excluding, or repeating attention blocks within the 32-layer architecture. The repetition strategy was particularly attractive as it could theoretically reduce memory footprint by reusing the same layer weights multiple times rather than storing distinct parameters for each position. This weight sharing approach aligned directly with the target hardware constraints outlined in Section \ref{target_hardware}, where the memory limitation makes parameter reduction a critical optimization target.

25 different layer configurations were tested, each of which was evaluated through qualitative text generation tasks and quantitative assessment on the HellaSwag dataset \cite{hellaswag}.
The results revealed several that conservative modification, often maintained reasonable performance while reducing computational overhead. In particular, skipping the layers more towards the middle of the model rather than its ends resulted in low performance degradation. For instance, the configuration that skipped layers 23-27 achieved a HellaSwag score of 0.38 compared to the baseline's 0.34, suggesting that certain middle layers may contribute less to final performance than expected.

However, more aggressive modifications typically led to severe degradation in output quality. Configurations involving extensive layer repetition or using only sparse layer selections often produced incoherent text with non-ASCII characters and semantic breakdown. This behavior indicated that while some redundancy exists in the transformer architecture, maintaining a balanced representation across different depths remains crucial for coherent language generation.

These preliminary findings informed the subsequent approach to systematic compression: it revealed that strategic layer removal could sometimes improve performance metrics, suggesting that pruning techniques might offer promising avenues for optimization; at the same time, it showed how layer repetition was not a viable strategy and caused heavy performance degradation.

\section{An Overview of the Pipeline}

The core result of this research is the compression pipeline, which implements a sequential approach that combines multiple optimization techniques in a carefully orchestrated manner. As previously mentioned, the target model for these optimizations is LLaMA 3.2 1B, a distilled version of the larger LLaMA 3.2 model that has already undergone some level of pruning during its creation, according to the Hugging Face model documentation \cite{llama3_1b}.

The choice of this particular model as the starting point is both strategic and practical. While incorporating distillation directly into the pipeline would have been ideal given its effectiveness as a compression technique, the computational requirements make it rather prohibitive. Distillation essentially requires training a model from scratch, demanding extensive GPU resources that far exceed the computational budget available for the project. Thus, an existing pre-distilled model was utilized instead which already provides a compact yet capable foundation.

The pipeline follows a five-stage progression, with each stage building upon the previous one. The particular ordering was chosen based on the complementary nature of these techniques and their relative impact on model structure.

\begin{enumerate}
    \item The first stage implements depth-wise pruning (Section \ref{depth_pruning}), where entire transformer layers are removed based on importance metrics computed during a calibration phase. This coarse-grained approach eliminates redundant blocks, and ultimately redefines the skeleton of the architecture.

    \item Width-wise pruning follows as the second stage (Section \ref{wanda}), applying the WANDA algorithm to remove less critical weights within the remaining layers. This is a finer-grained approach compared to the depth-pruning, as it operates at the parameter level.

    \item The third stage introduces \textit{Low-Rank Adaptation} (LoRA) (Section \ref{lora}) to recover performance lost during the pruning phases, while also allowing for fine-tuning on specific downstream tasks.

    \item The fourth stage applies 4-bit quantization using GPTQ, reducing the memory footprint of individual parameters.

    \item The final stage includes an optional \textit{Eigenspace Low-Rank Approximation} (EoRA) step to improve the performance of the quantized model.
\end{enumerate}

Detailed descriptions of each stage are provided in the following sections. Each step yields an intermediate model suitable for independent evaluation, except when explicitly stated otherwise. The modular design supports selective execution through comprehensive tuning options, such as the ability to skip specific stages or experiment with different parameter configurations without modifying the core implementation. Having intermediate models allows to conduct ablation studies more easily and adapt the pipeline to specific hardware constraints.

\section{Depth-wise Pruning} \label{depth_pruning}

Depth pruning represents a structured approach to neural network compression that targets the architectural dimension of model depth rather than individual parameter elimination. Unlike width pruning, which reduces the size of weight matrices by removing neurons or attention heads while preserving the total number of layers, depth pruning removes entire layers or blocks from the network architecture. In the context of transformer-based language models, this typically involves eliminating complete attention blocks, each containing both a multi-head attention module and feed-forward network components. Figure \ref{fig:pruning_comparison} shows a visual comparison between the two pruning techniques.

\begin{figure}[htbp]
    \centering
    \includegraphics[width=0.7\textwidth]{pruning.png}
    \caption[Comparison of Depth and Width Pruning]{In depth pruning, we remove the single transformer layers (left). On the other hand, in width pruning we remove single neurons from the weight matrices (right). The image was sourced from \cite{shortened_llama}.}
    \label{fig:pruning_comparison}
\end{figure}

The depth pruning approach follows the methodology established by Kim et al. in their Shortened LLaMA work \cite{shortened_llama}, which demonstrated that removing entire transformer blocks can achieve competitive performance while delivering significant inference speedups, particularly under memory-constrained conditions. Rather than reducing individual weight dimensions as in width pruning, depth pruning eliminates entire layers from the model architecture, creating a more direct path to computational savings.

The depth pruning process begins with calculating layer importance scores using one of several metrics implemented in the pipeline. Three primary importance calculation methods are supported: magnitude-based scoring, Taylor expansion-based gradient analysis, and perplexity-based evaluation. The magnitude approach computes the L1 norm of weights within each transformer block, providing a simple baseline for layer significance. The Taylor method leverages first-order gradient information to estimate the impact of layer removal on model performance, following the approximation $L(W = 0) \approx L(W) + \nabla L \cdot (-W)$ where the gradient-weight product indicates layer importance. The perplexity method evaluates each layer by temporarily removing it and measuring the resulting degradation in language modeling performance on a calibration dataset.

Following the insights from Shortened LLaMA, the implementation protects the first four and last two layers from pruning, as these positions have been shown to be critical for maintaining performance (which is in line with what was discovered in Section \ref{frankenllama}). The remaining layers are ranked according to their importance scores, with the least significant layers selected for removal.

The removal process involves creating a new model architecture with the selected layers entirely eliminated, maintaining the original transformer structure for the remaining blocks. This approach contrasts sharply with width pruning techniques that create sparse weight matrices, as depth pruning produces models with clean, dense architectures that map naturally onto hardware accelerators without requiring specialized sparse computation support.

The implementation extends the Shortened LLaMA approach by integrating it into a comprehensive compression pipeline, where depth pruning serves as the foundation for subsequent optimization techniques. The layer importance calculations are performed efficiently using calibration datasets, typically requiring only 10 samples from standard corpora to produce reliable importance rankings. This efficiency makes depth pruning particularly attractive for resource-constrained scenarios where extensive calibration is impractical.

\section{Width-wise Pruning} \label{wanda}

Following depth-wise pruning, the second stage of the compression pipeline applies width-wise pruning. As mentioned in Section \ref{depth_pruning}, it operates at the parameter level, eliminating individual weights within the remaining layers to achieve fine-grained compression. Width pruning can be either:
\begin{itemize}
    \item \textbf{Unstructured}:
    \item \textbf{Structured}:
\end{itemize}

\subsection{WANDA Algorithm Overview}

WANDA introduces a novel pruning metric that combines weight magnitudes with input activation statistics to determine parameter importance. The core insight behind this approach stems from the observation that large language models exhibit emergent large magnitude features in their hidden states—outlier activations that are significantly larger than typical values and are crucial for model performance.

The WANDA importance metric for each weight $W_{ij}$ is defined as:

\begin{equation}
S_{ij} = |W_{ij}| \cdot \|X_j\|_2
\end{equation}

where $|W_{ij}|$ represents the absolute value of the weight and $\|X_j\|_2$ evaluates the L2 norm of the $j$-th input feature across all calibration samples. This formulation addresses a key limitation of traditional magnitude-based pruning: it fails to account for input activations, which play an equally important role in determining neuron outputs, especially in the presence of large magnitude features.

\subsection{Structured N:M Sparsity Implementation}

While WANDA was originally designed for unstructured sparsity, the compression pipeline implements its structured variant to leverage hardware acceleration capabilities. Specifically, the implementation focuses on structured N:M sparsity patterns, where at most N out of every M consecutive weights are non-zero.

The structured pruning process operates as follows:

\begin{enumerate}
   \item \textbf{Importance Calculation}: For each linear layer, compute the WANDA metric for all weights using calibration data
   \item \textbf{Grouping}: Organize weights into groups of M consecutive elements within each output channel
   \item \textbf{Selection}: Within each group, identify the N weights with the highest importance scores
   \item \textbf{Pruning}: Set the remaining (M-N) weights to zero
\end{enumerate}

This structured approach enables the use of specialized hardware accelerators, such as NVIDIA's sparse tensor cores, which can efficiently handle N:M sparsity patterns during inference.

\subsection{Implementation Details}

The WANDA implementation integrates seamlessly into the compression pipeline through the \texttt{prune\_wanda} function. Key implementation aspects include:

\textbf{Calibration Setup}: The algorithm uses the same calibration dataset as the depth pruning stage, typically consisting of 10 samples from the C4 corpus, ensuring consistency across optimization stages.

\textbf{Layer-wise Processing}: WANDA processes each transformer layer sequentially, updating activations after each layer is pruned to ensure that subsequent layers receive realistic input distributions.

\textbf{Per-output Comparison}: Following the original WANDA methodology, weights are compared and pruned on a per-output basis rather than globally across the layer. This approach maintains balanced pruning ratios across different output features, which proves crucial for preserving model performance.

\textbf{Binary Search for Sparsity}: For the variant implementation, a binary search algorithm finds the optimal threshold parameter $\alpha$ that achieves the target sparsity ratio while maximizing the cumulative importance of retained weights.

\subsection{Advantages and Computational Efficiency}

WANDA offers several advantages over more complex pruning methods:

\begin{itemize}
   \item \textbf{Single Forward Pass}: Unlike methods requiring iterative weight updates, WANDA completes pruning in a single forward pass through the model
   \item \textbf{No Weight Updates}: The algorithm requires no modifications to the remaining weights, suggesting that effective sparse subnetworks exist within the original model
   \item \textbf{Low Computational Overhead}: With $O(d^2)$ complexity compared to $O(d^3)$ for second-order methods, WANDA provides significant computational savings
   \item \textbf{Hardware Compatibility}: The structured N:M variant directly maps to accelerator capabilities without requiring specialized sparse computation libraries
\end{itemize}

The integration of WANDA as the width pruning stage creates a foundation for subsequent optimization techniques while maintaining the model's core functionality. The structured sparsity patterns ensure that the compressed model can benefit from hardware acceleration during inference, making this approach particularly suitable for deployment scenarios with strict performance requirements.
\section{LoRA} \label{lora}
\section{Quantization and EoRA} \label{quantization}
\section{Evaluation}

\chapter{Experimental Results and Analysis}
\label{chap:results}
\lhead[\fancyplain{}{\bfseries\thepage}]{\fancyplain{}{\bfseries\rightmark}}
intro here

\section{Preliminary observation of the results}
Before examining the results based on In this section, we will present and comment on some examples of text generated by the 
\section{Results on TriviaQA}
\section{Results on WikiText}

\chapter{Conclusion and Future Work}
\label{chap:conclusion}
\lhead[\fancyplain{}{\bfseries\thepage}]{\fancyplain{}{\bfseries\rightmark}}
% Chapter 5: Conclusions

The compression pipeline developed throughout this thesis represents a comprehensive approach to reducing the computational and memory requirements of large language models while maintaining acceptable performance levels. By combining multiple optimization techniques in a carefully orchestrated sequence, the work demonstrates that aggressive model compression remains viable even for already-distilled architectures like LLaMA 3.2 1B.

The sequential application of depth pruning, width pruning, LoRA adaptation, quantization, and EoRA compensation creates a systematic pathway from the original 1B parameter model to significantly more compact variants suitable for deployment on resource-constrained hardware.

The evaluation framework reveals that these techniques, when properly integrated, can achieve substantial compression ratios while preserving core language modeling capabilities. The WikiText-2 perplexity results demonstrate that compressed models retain their fundamental ability to predict token sequences, while TriviaQA evaluations confirm that both factual knowledge retention and reading comprehension abilities survive the compression process with manageable degradation for moderate compression ratios, and can often be restored using a LoRA adapter in configurations where heavy performance loss occurs.

However, the work also highlights several limitations that constrain its immediate applicability. The most significant challenge lies in the substantial deterioration observed under aggressive compression ratios, where even refinement techniques such as LoRA adaptation and EoRA compensation cannot fully recover the capabilities lost during extensive pruning, resulting in models that struggle to maintain coherent language generation and factual accuracy for the most memory-constrained edge devices. Additionally, the sequential nature of the pipeline may not capture optimal interactions between different compression stages. The evaluation framework also presents limitations, as assessments concentrated primarily on English language tasks and, despite using relatively large prompts for reading comprehension tasks, focused on relatively short-context scenarios compared to the extended sequences that modern LLMs are increasingly expected to handle, leaving uncertainty about compression behavior on longer sequences and multilingual applications.

Nevertheless, this work establishes a solid foundation for practical model compression and demonstrates that multi-stage optimization approaches leveraging pruning techniques and quantization represent a viable pathway toward developing both environmentally sustainable LLMs and making advanced AI capabilities accessible across low-powered, resource-constrained devices.

\section{Future Work} \label{future_work}

While the models produced by the compression pipeline developed in this thesis demonstrate promising results across multiple evaluation benchmarks, numerous avenues remain unexplored that could significantly enhance both the quality of compressed models and the scope of the framework itself. These directions encompass methodological refinements, algorithmic innovations, and broader architectural compatibility considerations that could substantially improve the project's capabilities and applicability.

\subsection{Knowledge Distillation Integration}

One of the most significant limitations of the current approach lies in its reliance on pre-distilled models rather than incorporating distillation directly into the compression pipeline. As outlined in Section \ref{distillation_paragraph}, knowledge distillation represents a rather effective compression technique, yet computational constraints prevented its integration into this work. Future research should prioritize developing distillation workflows that can operate within reasonable computational budgets.

In this context, an interesting direction could involve adopting approaches such as \textit{Homotopic Distillation} (HomoDistil) \cite{homodistil}, which combines iterative pruning with knowledge distillation in a unified framework. HomoDistil (Section \ref{distillation_paragraph}) addresses the capacity gap problem by starting with the full teacher model and gradually removing neurons while simultaneously distilling knowledge, maintaining small prediction discrepancies throughout the process. This approach could be particularly well-suited to the current pipeline, as it directly addresses the performance degradation observed under aggressive pruning by providing continuous guidance during the compression process rather than attempting recovery afterward.

\subsection{Additional Quantization Methodologies}

While GPTQ proved effective in the current pipeline, the quantization landscape continues evolving rapidly with new algorithms which address different aspects of the precision-accuracy trade-off. \textit{Activation-aware Weight Quantization} (AWQ) \cite{awq} represents a particularly promising alternative that identifies salient weight channels based on activation distributions rather than weight magnitudes alone. AWQ's key insight that protecting only 1\% of the most important weights can dramatically reduce quantization error suggests potential for even more aggressive compression than achieved with GPTQ. Unlike GPTQ's reliance on Hessian information, AWQ employs mathematically equivalent transformations to scale salient channels, avoiding hardware-inefficient mixed-precision schemes while maintaining better generalization across domains.

\textit{Quality Quattuor-bit Quantization} (QQQ) \cite{qqq} offers another compelling approach that addresses the performance-speed trade-off through W4A8 quantization (4-bit weights, 8-bit activations). The method combines two key innovations: adaptive smoothing that selectively targets activation channels with significant outliers while preserving other channels, and Hessian-based compensation that employs the same mathematical framework as GPTQ for iterative weight adjustments to minimize quantization losses. This dual approach enables QQQ to handle both weight and activation quantization simultaneously, potentially achieving better compression ratios than weight-only methods while maintaining acceptable results. Notably, QQQ is natively supported by the GPTQModel toolkit \cite{gptqmodel}, making integration into the existing pipeline straightforward and potentially offering superior performance recovery compared to current weight-only quantization approaches.

\subsection{Advanced Investigations of Interdependencies} \label{advanced_interdep}

The current pipeline implements compression techniques sequentially with minimal consideration of their interactions, yet understanding how different stages influence each other could provide insights into significantly improved compression strategies. Future investigations should systematically examine adaptive methodologies that adjust later techniques based on earlier results, following a similar approach to 2SSP \cite{2ssp} (Section \ref{pruning_paragraph_chap2}); for instance, the ratios and thresholds for width pruning could be dynamically determined based on the specific layers removed during depth pruning, potentially targeting different sparsity levels in regions where architectural changes have already occurred.

Likewise, applying consistent methodologies across compression stages warrants exploration, such as using perplexity-based importance for both depth and width pruning rather than the current mixed approach of perplexity for layers and magnitude-based metrics for weights. The layer preservation employed in depth pruning, which safeguards the first four and last two layers from removal, could also be extended to width pruning operations. This would protect critical architectural components across both compression dimensions while enabling more aggressive optimization in intermediate regions.

The unification of recovery methods represents another possible improvement. Rather than applying LoRA and EoRA as separate sequential steps, these techniques could potentially be unified into a single adapter mechanism, in order to simplify the recovery process by eliminating intermediate steps.

Otherwise, EoRA can be leveraged as superior starting point for LoRA fine-tuning. The original EoRA paper \cite{eora} demonstrates that using EoRA matrices to initialize LoRA substantially enhances accuracy recovery compared to standard methods. This strategy could prove particularly effective for heavily compressed models, where EoRA's eigenspace projection provides a more informed foundation for subsequent gradient-based optimization.

Alternative sequencing configurations also warrant systematic investigation. Rather than the current Depth$\rightarrow$Width$\rightarrow$LoRA progression, arrangements such as Depth$\rightarrow$LoRA$\rightarrow$Width could allow for performance recovery immediately after the most aggressive structural changes, potentially preserving more model capabilities during subsequent fine-grained pruning\footnote{Preliminary experiments on this Depth$\rightarrow$LoRA$\rightarrow$Width configuration were conducted, with results presented in Appendix \ref{app:appendix2}.}.

\subsection{Extended Model Compatibility}
The current pipeline's focus on LLaMA architectures, while strategic for this work, limits its broader applicability. Extending support to popular open-weight model families like DeepSeek \cite{deepseek} and Qwen \cite{qwen} would significantly increase the framework's utility and enable broader comparative studies across different architectural paradigms.

While some components of the pipeline already support these architectures through underlying libraries like GPTQModel, which provides native support for both DeepSeek and Qwen models \cite{gptqmodel}, other aspects of the framework may not accommodate these changes seamlessly. The compression pipeline's layer importance calculations, pruning strategies, and evaluation protocols were specifically designed around LLaMA's architectural characteristics and may require substantial modifications for different model families. Additionally, availability constraints affect the practicality of extending to certain architectures: DeepSeek models are not available in small language model configurations comparable to LLaMA 3.2 1B, limiting their suitability for the target deployment scenarios. Conversely, Qwen offers promising alternatives with 1B and even 0.5B parameter variants in the Qwen2 series \cite{qwen2}, potentially enabling exploration of even more aggressive compression ratios than achievable with LLaMA.

Beyond expanding to existing architectures, the pipeline design should anticipate future model developments. Creating modular, architecture-agnostic compression components that can be easily adapted to new Transformer variants would ensure the framework's longevity.

Finally, one must consider that different architectures may exhibit varying sensitivities to compression techniques, requiring architecture-specific tuning of pruning criteria, quantization parameters, and LoRA configurations. Systematic studies of how compression techniques interact with different architectural choices could inform more effective optimization strategies and reveal whether the current pipeline's assumptions hold across diverse model families.

\subsection{Evaluation Framework Improvements}

The current evaluation methodology provides solid insights into compression effects on language model performance, yet several expansions can applied to increase the scope and depth of future assessments.

Extending the linguistic scope presents an opportunity to explore compression behavior across diverse language families. Different idioms exhibit varying degrees of morphological complexity that could interact differently with the pipeline, particularly when comparing agglutinative systems to isolating ones. Large-scale multilingual datasets like HPLT \cite{hplt}, covering 75 distinct tongues, offer promising avenues for comprehensive cross-linguistic evaluation and could reveal language-specific  vulnerabilities or robustness patterns. Additionally, it would be interesting to experiment with how LoRA adaptation on different languages from HPLT would influence performance on the WikiText-2 and TriviaQA evaluation methods previously adopted in this work.

Incorporating longer context evaluation represents another valuable enhancement. The focus on relatively short-context scenarios fails to capture compression behavior on larger sequences that increasingly define modern LLM applications. Thus, understanding how compression affects performance on sequences spanning thousands of tokens becomes essential. Extended context evaluation would be particularly valuable for compressed models, as the interaction between reduced parameter counts and long-range dependencies may create failure modes not fully visible in shorter evaluation scenarios.

The diversity of task assessments could be broadened. The current focus on perplexity and reading comprehension does not address the broad spectrum of capabilities that modern LLMs possess. Code generation represents a particularly valuable extension, as programming tasks require precise logical reasoning and structured output generation that may be disproportionately affected by parameter reduction. Recent proposals like CodeJudge \cite{codejudge} show that LLM-based evaluation can assess semantic correctness of generated code without requiring traditional test cases, and thus could be incorporated more easily into the evaluation framework.

\subsection{KV Cache Compression}
Key-value cache compression represents a critical bottleneck for auto-regressive generation that the current pipeline does not address. During inference, the KV cache stores attention keys and values for all previous tokens to prevent re-computation, but its size grows linearly with sequence length, creating substantial memory pressure that often exceeds hardware limitations \cite{kvcompr}.

Numerous techniques exist for KV cache compression, including quantization methods that reduce the precision of stored key-value pairs and pruning approaches that selectively remove components based on their significance \cite{kvcompr2}. Similar to how the current pipeline applies importance-based pruning to model parameters, KV cache optimizations can exploit fine-grained differences in significance across multiple dimensions (e.g. the differing computational impact of keys versus values in attention mechanisms, or the varying importance of individual tokens based on their contribution to subsequent predictions). Rather than applying uniform compression to all cache components, these approaches can selectively preserve the most critical elements while aggressively compressing or removing less important ones.

Recent frameworks like LeanKV \cite{kvcompr2} demonstrate that such differentiated approaches can achieve substantial compression ratios while maintaining near-lossless accuracy on complex reasoning tasks. Integrating similar KV cache optimization techniques into the pipeline could further optimize memory consumption beyond the current parameter reduction focus, and result in a more comprehensive approach that addresses both static model size and dynamic inference memory requirements.

\subsection{Engineering Improvements} \label{sec:future_work_engineering}
Several implementation enhancements could improve the practical performance of the compression pipeline without requiring algorithmic innovations. Flash Attention \cite{flash_attention} and its variants represent a direct optimization opportunity, as these techniques reorganize attention computation to minimize memory transfers between GPU memory hierarchies while maintaining mathematical equivalence to standard attention. Since Flash Attention operates independently of model compression, integrating it into the inference workflow would complement the existing compression techniques by reducing runtime memory pressure during the pipeline execution and model assessment phases without affecting model weights or architecture. This could be implemented through Hugging Face's BetterTransformer \cite{bettertransformer}, which provides optimized attention kernels including Flash Attention for supported architectures. However, BetterTransformer compatibility remains limited to specific model families, potentially creating challenges when extending the pipeline to Transformer variants beyond LLaMA 2 and 3, where such optimizations may not be readily available.

Supporting hardware-accelerated structured sparsity represents another feature that could significantly improve the pipeline's evaluation performance for models pruned in a structured fashion. As described in Section \ref{wanda}, NVIDIA's Ampere and Hopper GPU architectures provide native support for structured sparse matrix operations that can achieve theoretical 2$\times$ speedups over dense computations \cite{nvidia-width}. While the current pipeline produces models with structured sparsity patterns, the inference implementation does not exploit these hardware capabilities. Adapting the framework to leverage these acceleration features would require substantial compatibility efforts, as structured sparsity support remains in PyTorch's nightly builds rather than stable releases at the time of writing \cite{pytorch_sparsity}. In addition, implementation would likely necessitate redefining the Transformer architecture using the TorchAO optimization library \cite{torchao}, which provides the necessary primitives for structured sparse operations but requires careful integration with existing inference workflows.

Both improvements focus on optimizing the development and evaluation workflow rather than advancing compression methodologies themselves. While they could substantially enhance the practical utility and efficiency of the pipeline's operation, they do not necessarily translate to other embedded architectures where the models may be deployed, instead requiring specific hardware-dependent implementations that may not be available across all target platforms.

\begin{thebibliography}{9}
	\addcontentsline{toc}{chapter}{References}

	\bibitem{academic_integrity}
	Mike Perkins,
	\textit{Academic Integrity considerations of AI Large Language Models in the post-pandemic era: ChatGPT and beyond},
	\url{https://www.researchgate.net/publication/368775737_Academic_integrity_considerations_of_AI_Large_Language_Models_in_the_post-pandemic_era_ChatGPT_and_beyond}

	\bibitem{artistic_integrity}
	Daniel Mügge,
	\textit{AI Is Threatening More Than Just Creative Jobs—It’s Undermining Our Humanity},
	\url{https://www.socialeurope.eu/ai-is-threatening-more-than-%
	just-creative-jobs-its-undermining-our-humanity}

	\bibitem{gpt_energy}
	David Patterson, Joseph Gonzalez, Quoc Le, Chen Liang, Lluis-Miquel Munguia, Daniel Rothchild, David So, Maud Texier, Jeff Dean,
	\textit{Carbon Emissions and Large Neural Network Training},
	\url{https://arxiv.org/abs/2104.10350}

	\bibitem{datacenter_energy}
	Thomas Spencer, Siddharth Singh,
	\textit{What the data centre and AI boom could mean for the energy sector},
	\url{https://www.iea.org/commentaries/what-the-data-centre-and-ai-boom-could-mean-for-the-energy-sector}

	\bibitem{hungry_ai}
	Nidhal Jegham, Marwen Abdelatti, Lassad Elmoubarki, Abdeltawab Hendawi,
	\textit{How Hungry is AI? Benchmarking Energy, Water, and Carbon Footprint of LLM Inference}
	\url{https://arxiv.org/abs/2505.09598v1}

	\bibitem{google_report}
	Google,
	\textit{Google Environmental Report 2023},
	\url{https://sustainability.google/reports/google-2023-environmental-report-executive-summary/}

	\bibitem{ai_water}
	Pengfei Li, Jianyi Yang, Mohammad A. Islam, Shaolei Ren,
	\textit{Making AI Less "Thirsty": Uncovering and Addressing the Secret Water Footprint of AI Models},
	\url{https://arxiv.org/abs/2304.03271}

	\bibitem{gpt}
	Alec Radford, Karthik Narasimhan, Tim Salimans, Ilya Sutskever,
	\textit{Improving Language Understanding by Generative Pre-Training},
	\url{https://cdn.openai.com/research-covers/language-unsupervised/language_understanding_paper.pdf}

	\bibitem{gpt4}
	OpenAI,
	\textit{GPT-4 is OpenAI’s most advanced system, producing safer and more useful responses},
	\url{https://openai.com/index/gpt-4/}

	\bibitem{rnn}
	Chris Nicholson,
	\textit{A Beginner’s Guide to LSTMs and Recurrent Neural Networks}
	\url{https://skymind.ai/wiki/lstm}

	\bibitem{lstm}
	S. Hochreiter, J. Schmidhuber,
	\textit{Long Short-Term Memory},
	\url{https://doi.org/10.1162/neco.1997.9.8.1735}

	\bibitem{hinton-lstm}
	Alex Graves, Abdel-rahman Mohamed, Geoffrey Hinton,
	\textit{Speech Recognition with Deep Recurrent Neural Networks},
	\url{https://arxiv.org/abs/1303.5778}

	\bibitem{lstm_textgeneration}
	Mustafa Abbas Hussein Hussein, Serkan Savaş,
	\textit{LSTM-Based Text Generation: A Study on Historical Datasets},
	\url{https://arxiv.org/abs/2403.07087}

	\bibitem{bpe}
	Philip Gage,
	\textit{A New Algorithm for Data Compression},
	\url{http://www.pennelynn.com/Documents/CUJ/HTML/94HTML/19940045.HTM}

	\bibitem{attention_is_all_you_need}
	Ashish Vaswani, Noam Shazeer, Niki Parmar, Jakob Uszkoreit, Llion Jones, Aidan N. Gomez, Lukasz Kaiser, Illia Polosukhin,
	\textit{Attention is All You Need},
	\url{https://arxiv.org/abs/1706.03762}

	\bibitem{llama}
	Hugo Touvron, Thibaut Lavril, Gautier Izacard, Xavier Martinet, Marie-Anne Lachaux, Timothée Lacroix, Baptiste Rozière, Naman Goyal, Eric Hambro, et al.,
	\textit{LLaMA: Open and Efficient Foundation Language Models},
	\url{https://arxiv.org/abs/2302.13971}

	\bibitem{llama2}
	Hugo Touvron, Louis Martin, Kevin Stone, Peter Albert, Amjad Almahairi, Yasmine Babaei, Nikolay Bashlykov, Soumya Batra, Prajjwal Bhargava, et al.,
	\textit{Llama 2: Open Foundation and Fine-Tuned Chat Models},
	\url{https://arxiv.org/abs/2307.09288}

	\bibitem{llama3}
	Llama Team, AI @ Meta,
	\textit{The Llama 3 Herd of Models},
	\url{https://arxiv.org/abs/2407.21783}

	\bibitem{target_hardware}
	Arpan Suravi Prasad, Moritz Scherer, Francesco Conti, Davide Rossi, Alfio Di Mauro, Manuel Eggimann, Jorge Tómas Gómez, Ziyun Li, Syed Shakib Sarwar, Zhao Wang, Barbara De Salvo, Luca Benini,
	\textit{Siracusa: A 16 nm Heterogenous RISC-V SoC for Extended Reality with At-MRAM Neural Engine},
	\url{https://arxiv.org/abs/2312.14750}
	
	\bibitem{distillation}
	Geoffrey Hinton, Oriol Vinyals, Jeff Dean,
	\textit{Distilling the Knowledge in a Neural Network},
	\url{https://arxiv.org/abs/1503.02531}
	
	\bibitem{homodistil}
	Chen Liang, Haoming Jiang, Zheng Li, Xianfeng Tang, Bin Yin, Tuo Zhao,
	\textit{HomoDistil: Homotopic Task-Agnostic Distillation of Pre-trained Transformers},
	\url{https://arxiv.org/abs/2302.09632}

	\bibitem{2ssp}
	Fabrizio Sandri, Elia Cunegatti, Giovanni Iacca,
	\textit{2SSP: A Two-Stage Framework for Structured Pruning of LLMs},
	\url{https://arxiv.org/abs/2501.17771}

	\bibitem{sheared_llama}
	Mengzhou Xia, Tianyu Gao, Zhiyuan Zeng, Danqi Chen,
	\textit{Sheared LLaMA: Accelerating Language Model Pre-training via Structured Pruning},
	\url{https://arxiv.org/abs/2310.06694}

	\bibitem{franken-llama}
	Angelo Galavotti,
	\textit{FRANKEN-LLAMA code repository},
	\url{https://github.com/AngeloGalav/franken-llama}

	\bibitem{pytorch}
	Adam Paszke, Sam Gross, Francisco Massa, Adam Lerer, James Bradbury, Gregory Chanan, Trevor Killeen, Zeming Lin, Natalia Gimelshein, et al.,
	\textit{PyTorch: An Imperative Style, High-Performance Deep Learning Library},
	\url{https://arxiv.org/abs/1912.01703}

	\bibitem{hf_transformers}
	Hugging Face,
	\textit{Transformers library documentation},
	\url{https://huggingface.co/docs/transformers/index}

	\bibitem{hellaswag}
	Rowan Zellers, Ari Holtzman, Yonatan Bisk, Ali Farhadi, Yejin Choi,
	\textit{HellaSwag: Can a Machine Really Finish Your Sentence?}
	\url{https://arxiv.org/abs/1905.07830}

	\bibitem{llama3_1b}
	Llama Team, AI @ Meta,
	\textit{Llama-3.2-1B},
	\url{https://huggingface.co/meta-llama/Llama-3.2-1B}

	\bibitem{llama3_1b_instruct}
	Llama Team, AI @ Meta,
	\textit{Llama-3.2-1B-Instruct},
	\url{https://huggingface.co/meta-llama/Llama-3.2-1B-Instruct}

	\bibitem{shortened_llama}
	Bo-Kyeong Kim, Geonmin Kim, Tae-Ho Kim, Thibault Castells, Shinkook Choi, Junho Shin, Hyoung-Kyu Song,
	\textit{Shortened LLaMA: Depth Pruning for Large Language Models with Comparison of Retraining Methods},
	\url{https://arxiv.org/abs/2402.02834}

	\bibitem{wanda}
	Mingjie Sun, Zhuang Liu, Anna Bair, J. Zico Kolter,
	\textit{A Simple and Effective Pruning Approach for Large Language Models},
	\url{https://arxiv.org/abs/2306.11695}

	\bibitem{nvidia-width}
	Hongxiao Bai, Yun Li,
	\textit{Structured Sparsity in the NVIDIA Ampere Architecture and Applications in Search Engines},
	\url{https://developer.nvidia.com/blog/structured-sparsity-in-the-nvidia-ampere-architecture-and-applications-in-search-engines/}

	\bibitem{lora}
	Edward J. Hu, Yelong Shen, Phillip Wallis, Zeyuan Allen-Zhu, Yuanzhi Li, Shean Wang, Lu Wang, Weizhu Chen,
	\textit{LoRA: Low-Rank Adaptation of Large Language Models},
	\url{https://arxiv.org/abs/2106.09685}

	\bibitem{peft}
	Hugging Face,
	\textit{PEFT: State-of-the-art Parameter-Efficient Fine-Tuning},
	\url{https://github.com/huggingface/peft}

	\bibitem{quant_cnn}
	Jiaxiang Wu, Cong Leng, Yuhang Wang, Qinghao Hu, Jian Cheng,
	\textit{Quantized Convolutional Neural Networks for Mobile Devices},
	\url{https://arxiv.org/abs/1512.06473}

	\bibitem{gptq_quantization}
	Elias Frantar, Saleh Ashkboos, Torsten Hoefler, Dan Alistarh,
	\textit{GPTQ: Accurate Post-Training Quantization for Generative Pre-trained Transformers},
	\url{https://arxiv.org/abs/2210.17323}

	\bibitem{obq}
	Elias Frantar, Sidak Pal Singh, Dan Alistarh,
	\textit{Optimal Brain Compression: A Framework for Accurate Post-Training Quantization and Pruning},
	\url{https://arxiv.org/abs/2208.11580}

	\bibitem{gptqmodel}
	ModelCloud,
	\textit{GPTQModel},
	\url{https://github.com/ModelCloud/GPTQModel}

	\bibitem{c4}
	Colin Raffel, Noam Shazeer, Adam Roberts, Katherine Lee, Sharan Narang, Michael Matena, Yanqi Zhou, Wei Li, Peter J. Liu,
	\textit{Exploring the Limits of Transfer Learning with a Unified Text-to-Text Transformer},
	\url{https://arxiv.org/abs/1910.10683}

	\bibitem{eora}
	Shih-Yang Liu, Maksim Khadkevich, Nai Chit Fung, Charbel Sakr, Chao-Han Huck Yang, Chien-Yi Wang, Saurav Muralidharan, Hongxu Yin, Kwang-Ting Cheng, et al.,
	\textit{EoRA: Fine-tuning-free Compensation for Compressed LLM with Eigenspace Low-Rank Approximation},
	\url{https://arxiv.org/abs/2410.21271}

	\bibitem{wikitext}
	Stephen Merity, Caiming Xiong, James Bradbury, Richard Socher,
	\textit{Pointer Sentinel Mixture Models},
	\url{https://arxiv.org/abs/1609.07843}

	\bibitem{perplexity}
	F. Jelinek, R. L. Mercer, L. R. Bahl, J. K. Baker,
	\textit{Perplexity—a measure of the difficulty of speech recognition tasks},
	\url{https://doi.org/10.1121/1.2016299}

	\bibitem{triviaqa}
	Mandar Joshi, Eunsol Choi, Daniel S. Weld, Luke Zettlemoyer,
	\textit{TriviaQA: A Large Scale Distantly Supervised Challenge Dataset for Reading Comprehension},
	\url{https://arxiv.org/abs/1705.03551}

	\bibitem{qqq}
	Ying Zhang, Peng Zhang, Mincong Huang, Jingyang Xiang, Yujie Wang, Chao Wang, Yineng Zhang, Lei Yu, Chuan Liu, Wei Lin,
	\textit{QQQ: Quality Quattuor-Bit Quantization for Large Language Models},
	\url{https://arxiv.org/abs/2406.09904}

	\bibitem{awq}
	Ji Lin, Jiaming Tang, Haotian Tang, Shang Yang, Wei-Ming Chen, Wei-Chen Wang, Guangxuan Xiao, Xingyu Dang, Chuang Gan, Song Han,
	\textit{AWQ: Activation-aware Weight Quantization for LLM Compression and Acceleration},
	\url{https://arxiv.org/abs/2306.00978}

	\bibitem{mistral}
	Albert Q. Jiang, Alexandre Sablayrolles, Arthur Mensch, Chris Bamford, Devendra Singh Chaplot, Diego de las Casas, Florian Bressand, et al.,
	\textit{Mistral 7B},
	\url{https://arxiv.org/abs/2310.06825}

	\bibitem{deepseek}
	DeepSeek-AI, Aixin Liu, Bei Feng, Bing Xue, Bingxuan Wang, Bochao Wu, Chengda Lu, Chenggang Zhao, Chengqi Deng, Chenyu Zhang, et al.,
	\textit{DeepSeek-V3 Technical Report},
	\url{https://arxiv.org/abs/2412.19437}

	\bibitem{qwen}
	Jinze Bai, Shuai Bai, Yunfei Chu, Zeyu Cui, Kai Dang, Xiaodong Deng, Yang Fan, Wenbin Ge, Yu Han, Fei Huang, Binyuan Hui, Luo Ji, et al.,
	\textit{Qwen Technical Report},
	\url{https://arxiv.org/abs/2309.16609}

	\bibitem{qwen2}
	HuggingFace,
	\textit{Qwen's Collections},
	\url{https://huggingface.co/collections/Qwen/qwen2-6659360b33528ced941e557f}

	\bibitem{kvcompr}
	Jiayi Yuan, Hongyi Liu, Shaochen Zhong, Yu-Neng Chuang, Songchen Li, Guanchu Wang, Duy Le, Hongye Jin, Vipin Chaudhary, Zhaozhuo Xu, et al.,
	\textit{KV Cache Compression, But What Must We Give in Return? A Comprehensive Benchmark of Long Context Capable Approaches},
	\url{https://arxiv.org/abs/2407.01527}

	\bibitem{kvcompr2}
	Yanqi Zhang, Yuwei Hu, Runyuan Zhao, John C.S. Lui, Haibo Chen,
	\textit{Unifying KV Cache Compression for Large Language Models with LeanKV},
	\url{https://arxiv.org/abs/2412.03131v2}

	\bibitem{flash_attention}
	Tri Dao, Daniel Y. Fu, Stefano Ermon, Atri Rudra, Christopher Ré,
	\textit{FlashAttention: Fast and Memory-Efficient Exact Attention with IO-Awareness}
	\url{https://arxiv.org/abs/2205.14135}

	\bibitem{bettertransformer}
	HuggingFace,
	\textit{BetterTransformer Overview},
	\url{https://huggingface.co/docs/optimum/bettertransformer/overview}

	\bibitem{pytorch_sparsity}
	PyTorch,
	\textit{Accelerating Neural Network Training with Semi-Structured (2:4) Sparsity},
	\url{https://pytorch.org/blog/accelerating-neural-network-training/}

	\bibitem{torchao}
	PyTorch,
	\textit{ao: PyTorch native quantization and sparsity for training and inference},
	\url{https://github.com/pytorch/ao}


\end{thebibliography}

\appendix
\chapter{Detailed Experimental Data}
\label{app:appendix1}
This appendix provides comprehensive experimental results across all tested configurations, including intermediate compression stages and generation examples that illustrate qualitative performance changes.

\section{Pruning-Only Configurations} \label{app:appendix1_pruning}

This section examines models that have undergone only pruning operations (depth and width pruning) without additional optimization techniques.

\subsection{Benchmark Results}

{\footnotesize
\begin{longtable}{lcccccc}
\caption[Results for Pruning-Only Configurations]{Performance results of pruned models.} \label{tab:full_pipeline_results} \\
\hline
\textbf{Config} & \multicolumn{2}{c}{\textbf{TriviaQA (\%) $\uparrow$}} & & \textbf{WikiText $\downarrow$} & \textbf{\#Params} & \textbf{Size} \\
\cline{2-3}
& \textbf{Closed} & \textbf{Open} & & \textbf{PPL} & & \textbf{(GB)} \\
\hline
\endfirsthead

\multicolumn{7}{c}%
{{\bfseries \tablename\ \thetable{} -- continued from previous page}} \\
\hline
\textbf{Config} & \multicolumn{2}{c}{\textbf{TriviaQA (\%) $\uparrow$}} & & \textbf{WikiText $\downarrow$} & \textbf{\#Params} & \textbf{Size} \\
\cline{2-3}
& \textbf{Closed} & \textbf{Open} & & \textbf{PPL} & & \textbf{(GB)} \\
\hline
\endhead

\hline \multicolumn{7}{r}{{Continued on next page}} \\
\endfoot

\hline
\endlastfoot

\textit{Baseline Instruct} & \textit{50.6} & \textit{81.8} & & \textit{26.61} & \textit{1235.8M} & \textit{2.30} \\
Baseline Vanilla & 40.2 & 77.8 & & 20.05 & 1235.8M & 2.30 \\
Depth 2 & 20.0 & 53.0 & & 55.24 & 1114.2M & 2.08 \\
Depth 4 & 5.6 & 14.2 & & 230.77 & 992.5M & 1.85 \\
Depth 6 & 2.0 & 2.4 & & 2081.04 & 870.9M & 1.62 \\
Depth 8 & 1.2 & 1.4 & & 227734.90 & 749.2M & \textbf{1.40} \\
Width 1:2 & 1.8 & 2.2 & & 471.66 & 749.3M & 2.30 \\
Width 2:4 & 7.0 & 17.2 & & 161.10 & 749.3M & 2.30 \\
Width 3:4 & 0.4 & 0.2 & & 10734.84 & 506.0M & 2.30 \\
Width 1:8 & 47.8 & 80.8 & & 27.19 & 1114.2M & 2.30 \\
Width 4:8 & 12.6 & 42.4 & & 81.32 & 749.3M & 2.30 \\
Width 6:8 & 0.2 & 0.2 & & 3075.58 & 506.0M & 2.30 \\
Width 1:16 & \textbf{49.2} & \textbf{81.8} & & \textbf{26.77} & 1175.0M & 2.30 \\
Width 8:16 & 16.2 & 52.8 & & 64.08 & 749.3M & 2.30 \\
Width 12:16 & 0.2 & 0.2 & & 1793.97 & 506.0M & 2.30 \\
Depth 2 + Width 1:2 & 0.4 & 0.4 & & 694.34 & 688.4M & 2.08 \\
Depth 2 + Width 2:4 & 3.8 & 5.0 & & 309.32 & 688.4M & 2.08 \\
Depth 2 + Width 3:4 & 0.2 & 0.2 & & 38058.32 & 475.6M & 2.08 \\
Depth 2 + Width 1:8 & 20.2 & 53.6 & & 57.22 & 1007.7M & 2.08 \\
Depth 2 + Width 4:8 & 6.4 & 15.2 & & 185.43 & 688.4M & 2.08 \\
Depth 2 + Width 6:8 & 0.6 & 1.0 & & 18260.58 & 475.6M & 2.08 \\
Depth 2 + Width 1:16 & 19.6 & 51.6 & & 55.68 & 1061.0M & 2.08 \\
Depth 2 + Width 8:16 & 7.6 & 23.4 & & 139.97 & 688.4M & 2.08 \\
Depth 2 + Width 12:16 & 0.2 & 0.4 & & 2589.90 & 475.6M & 2.08 \\
Depth 4 + Width 1:2 & 0.6 & 0.2 & & 3124.02 & 627.6M & 1.85 \\
Depth 4 + Width 2:4 & 1.0 & 2.2 & & 1167.36 & 627.6M & 1.85 \\
Depth 4 + Width 3:4 & 0.8 & 0.2 & & 87799.62 & 445.2M & 1.85 \\
Depth 4 + Width 1:8 & 5.2 & 14.8 & & 244.69 & 901.3M & 1.85 \\
Depth 4 + Width 4:8 & 3.0 & 2.6 & & 774.60 & 627.6M & 1.85 \\
Depth 4 + Width 6:8 & 0.2 & 0.2 & & 63239.70 & 445.2M & 1.85 \\
Depth 4 + Width 1:16 & 4.6 & 14.6 & & 235.32 & 946.9M & 1.85 \\
Depth 4 + Width 8:16 & 2.6 & 6.2 & & 591.59 & 627.6M & 1.85 \\
Depth 4 + Width 12:16 & 1.0 & 0.6 & & 119073.95 & 445.2M & 1.85 \\
Depth 6 + Width 1:2 & 0.6 & 0.8 & & 19899.25 & 566.8M & 1.62 \\
Depth 6 + Width 2:4 & 1.0 & 1.0 & & 20371.15 & 566.8M & 1.62 \\
Depth 6 + Width 3:4 & 0.4 & 0.6 & & 648756.35 & 414.8M & 1.62 \\
Depth 6 + Width 1:8 & 2.0 & 1.6 & & 2376.63 & 794.9M & 1.62 \\
Depth 6 + Width 4:8 & 0.8 & 1.0 & & 77482.90 & 566.8M & 1.62 \\
Depth 6 + Width 6:8 & 0.2 & 0.2 & & 108418.06 & 414.8M & 1.62 \\
Depth 6 + Width 1:16 & 1.8 & 1.6 & & 2032.84 & 832.9M & 1.62 \\
Depth 8 + Width 1:2 & 0.4 & 0.2 & & 111859.62 & 506.0M &  \textbf{1.40} \\
Depth 8 + Width 2:4 & 0.6 & 0.4 & & 185871.81 & 506.0M &  \textbf{1.40} \\
Depth 8 + Width 3:4 & 0.4 & 0.4 & & 1044840.71 & \textbf{384.3M} & \textbf{1.40} \\
Depth 8 + Width 1:8 & 1.4 & 1.0 & & 224204.19 & 688.4M &  \textbf{1.40} \\
Depth 8 + Width 4:8 & 0.8 & 0.4 & & 554910.71 & 506.0M &  \textbf{1.40} \\
\end{longtable}
}
\normalsize

\section{Optimized Configurations} \label{app:appendix1_optimized}

This section presents the quantitative results for the configurations that have undergone stages of the compression pipeline after pruning, including LoRA adaptation, quantization, and EoRA compensation, as well as those that have completed the full pipeline.

\subsection{Benchmark Results}

\scriptsize
\begin{longtable}{lcclcccc}
\caption[Results for Complete Pipeline Configurations]{Evaluation results of the configurations subset to which LoRA, Quantization and Eora were applied.} \label{tab:all_results} \\
\hline
\textbf{Config} & \textbf{LoRA} & \textbf{Quant} & & \multicolumn{2}{c}{\textbf{TriviaQA (\%) $\uparrow$}} & \textbf{WikiText $\downarrow$} & \textbf{Size} \\
\cline{5-6}
& \textbf{Type} & & & \textbf{Closed} & \textbf{Open} & \textbf{PPL} & \textbf{(GB)} \\
\hline
\endfirsthead

\multicolumn{7}{c}%
{{\footnotesize \bfseries \tablename\ \thetable{} -- continued from previous page}} \\
\hline
\textbf{Config} & \textbf{LoRA} & \textbf{Quant} & & \multicolumn{2}{c}{\textbf{TriviaQA (\%) $\uparrow$}} & \textbf{WikiText $\downarrow$} & \textbf{Size} \\
\cline{5-6}
& \textbf{Type} & & & \textbf{Closed} & \textbf{Open} & \textbf{PPL} & \textbf{(GB)} \\
\hline
\endhead

\hline \multicolumn{8}{r}{{Continued on next page}} \\
\endfoot

\hline
\endlastfoot
\textit{Baseline Instruct} & \textit{None} & \textit{No} & & \textit{50.6} & \textit{81.8} & \textit{26.61} & \textit{2.30} \\
Depth 2 & WikiText & No & & 28.8 & 59.3 & 21.34 & 2.08 \\
Depth 2 & WikiText & Yes & & 23.2 & 53.5 & 22.89 & 0.92 \\
Depth 2 & WikiText & EoRA & & 24.7 & 57.9 & 22.63 & 0.92 \\
Depth 2 & TriviaQA & No & & 27.8 & 60.2 & 51.09 & 2.08 \\
Depth 2 & TriviaQA & Yes & & 23.9 & 54.6 & 55.03 & 0.92 \\
Depth 2 & TriviaQA & EoRA & & 24.4 & 58.4 & 54.60 & 0.92 \\
Depth 4 & WikiText & No & & 18.8 & 54.1 & 29.45 & 1.85 \\
Depth 4 & WikiText & Yes & & 15.3 & 49.3 & 31.60 & 0.86 \\
Depth 4 & TriviaQA & No & & 19.8 & 49.4 & 91.43 & 1.85 \\
Depth 4 & TriviaQA & Yes & & 17.3 & 41.4 & 103.61 & 0.86 \\
Depth 8 & WikiText & No & & 5.8 & 18.0 & 57.44 & 1.40 \\
Depth 8 & WikiText & Yes & & 4.7 & 18.0 & 61.87 & \textbf{0.74} \\
Depth 8 & WikiText & EoRA & & 5.0 & 17.5 & 59.50 & \textbf{0.74} \\
Depth 8 & TriviaQA & No & & 9.5 & 0.5 & 10005.97 & 1.40 \\
Depth 8 & TriviaQA & Yes & & 7.8 & 0.7 & 16888.28 & \textbf{0.74} \\
Depth 8 & TriviaQA & EoRA & & 9.5 & 0.4 & 7320.53 & \textbf{0.74} \\
Width 1:2 & WikiText & No & & 10.0 & 32.0 & 40.26 & 2.30 \\
Width 1:2 & WikiText & Yes & & 9.1 & 27.3 & 42.19 & 0.98 \\
Width 1:2 & WikiText & EoRA & & 8.8 & 28.3 & 41.78 & 0.98 \\
Width 1:2 & TriviaQA & No & & 12.4 & 29.3 & 145.54 & 2.30 \\
Width 1:2 & TriviaQA & Yes & & 11.3 & 26.5 & 154.33 & 0.98 \\
Width 1:2 & TriviaQA & EoRA & & 11.7 & 25.7 & 152.53 & 0.98 \\
Width 2:4 & WikiText & No & & 12.6 & 45.3 & 30.63 & 2.30 \\
Width 2:4 & WikiText & Yes & & 12.3 & 44.4 & 32.29 & 0.98 \\
Width 2:4 & TriviaQA & No & & 15.8 & 48.6 & 90.72 & 2.30 \\
Width 2:4 & TriviaQA & Yes & & 14.8 & 48.2 & 97.33 & 0.98 \\
Width 1:8 & WikiText & No & & \textbf{47.8} & \textbf{80.8} & \textbf{15.70} & 2.30 \\
Width 1:8 & WikiText & Yes & & 39.0 & 74.8 & 16.85 & 0.98 \\
Width 1:8 & WikiText & EoRA & & 40.0 & 78.4 & 16.68 & 0.98 \\
Width 1:8 & TriviaQA & No & & 44.9 & 79.4 & 30.81 & 2.30 \\
Width 1:8 & TriviaQA & Yes & & 41.3 & 77.2 & 33.38 & 0.98 \\
Width 1:8 & TriviaQA & EoRA & & 42.0 & 77.2 & 32.86 & 0.98 \\
Width 4:8 & WikiText & No & & 17.0 & 51.1 & 26.04 & 2.30 \\
Width 4:8 & WikiText & Yes & & 15.7 & 45.3 & 27.40 & 0.98 \\
Width 4:8 & TriviaQA & No & & 19.0 & 56.8 & 64.33 & 2.30 \\
Width 4:8 & TriviaQA & Yes & & 17.7 & 57.1 & 67.95 & 0.98 \\
Width 8:16 & WikiText & No & & 19.4 & 56.3 & 24.04 & 2.30 \\
Width 8:16 & WikiText & Yes & & 17.5 & 52.1 & 25.39 & 0.98 \\
Width 8:16 & WikiText & EoRA & & 17.8 & 54.9 & 25.09 & 0.98 \\
Width 8:16 & TriviaQA & No & & 20.6 & 57.6 & 56.11 & 2.30 \\
Width 8:16 & TriviaQA & Yes & & 20.8 & 55.7 & 60.67 & 0.98 \\
Width 8:16 & TriviaQA & EoRA & & 21.3 & 54.0 & 59.73 & 0.98 \\
Depth 2 + Width 1:16 & WikiText & No & & 28.1 & 61.8 & 21.42 & 2.08 \\
Depth 2 + Width 1:16 & WikiText & Yes & & 22.9 & 55.8 & 23.03 & 0.92 \\
Depth 2 + Width 1:16 & WikiText & EoRA & & 23.3 & 57.3 & 22.72 & 0.92 \\
Depth 2 + Width 1:16 & TriviaQA & No & & 28.0 & 59.5 & 50.99 & 2.08 \\
Depth 2 + Width 1:16 & TriviaQA & Yes & & 25.2 & 56.3 & 55.46 & 0.92 \\
Depth 2 + Width 1:16 & TriviaQA & EoRA & & 24.8 & 57.4 & 54.81 & 0.92 \\
Depth 2 + Width 1:8 & WikiText & No & & 26.5 & 63.8 & 21.76 & 2.08 \\
Depth 2 + Width 1:8 & WikiText & Yes & & 24.8 & 58.1 & 23.21 & 0.92 \\
Depth 2 + Width 1:8 & WikiText & EoRA & & 22.3 & 59.6 & 22.98 & 0.92 \\
Depth 2 + Width 1:8 & TriviaQA & No & & 27.4 & 60.4 & 51.89 & 2.08 \\
Depth 2 + Width 1:8 & TriviaQA & Yes & & 24.0 & 54.8 & 54.81 & 0.92 \\
Depth 2 + Width 1:8 & TriviaQA & EoRA & & 24.3 & 58.3 & 55.46 & 0.92 \\
Depth 2 + Width 2:4 & WikiText & No & & 11.5 & 31.3 & 39.33 & 2.08 \\
Depth 2 + Width 2:4 & WikiText & Yes & & 10.6 & 30.8 & 41.54 & 0.92 \\
Depth 2 + Width 2:4 & WikiText & EoRA & & 10.4 & 29.8 & 40.73 & 0.92 \\
Depth 2 + Width 2:4 & TriviaQA & No & & 11.1 & 29.6 & 146.68 & 2.08 \\
Depth 2 + Width 2:4 & TriviaQA & Yes & & 11.6 & 29.4 & 157.98 & 0.92 \\
Depth 2 + Width 2:4 & TriviaQA & EoRA & & 11.2 & 28.4 & 154.33 & 0.92 \\
Depth 2 + Width 3:4 & WikiText & No & & 1.2 & 5.3 & 134.60 & 2.08 \\
Depth 2 + Width 3:4 & WikiText & Yes & & 1.3 & 5.8 & 139.97 & 0.92 \\
Depth 2 + Width 3:4 & WikiText & EoRA & & 0.8 & 5.2 & 136.72 & 0.92 \\
Depth 2 + Width 3:4 & TriviaQA & No & & 2.1 & 0.5 & 3148.52 & 2.08 \\
Depth 2 + Width 3:4 & TriviaQA & Yes & & 2.3 & 0.7 & 3980.10 & 0.92 \\
Depth 2 + Width 3:4 & TriviaQA & EoRA & & 2.1 & 0.7 & 3431.06 & 0.92 \\
Depth 2 + Width 4:8 & WikiText & No & & 11.8 & 44.0 & 34.10 & 2.08 \\
Depth 2 + Width 4:8 & WikiText & Yes & & 10.6 & 42.6 & 35.81 & 0.92 \\
Depth 2 + Width 4:8 & TriviaQA & No & & 14.7 & 45.0 & 116.95 & 2.08 \\
Depth 2 + Width 4:8 & TriviaQA & Yes & & 13.8 & 43.0 & 124.98 & 0.92 \\
Depth 2 + Width 8:16 & WikiText & No & & 12.8 & 41.0 & 31.91 & 2.08 \\
Depth 2 + Width 8:16 & WikiText & Yes & & 12.3 & 42.0 & 33.51 & 0.92 \\
Depth 2 + Width 8:16 & TriviaQA & No & & 16.6 & 48.8 & 96.95 & 2.08 \\
Depth 2 + Width 8:16 & TriviaQA & Yes & & 15.3 & 46.2 & 103.61 & 0.92 \\
Depth 4 + Width 1:16 & WikiText & No & & 18.1 & 53.2 & 29.51 & 1.85 \\
Depth 4 + Width 1:16 & WikiText & Yes & & 15.2 & 48.6 & 31.66 & 0.86 \\
Depth 4 + Width 1:16 & TriviaQA & No & & 18.7 & 47.3 & 89.32 & 1.85 \\
Depth 4 + Width 1:16 & TriviaQA & Yes & & 16.8 & 39.3 & 100.42 & 0.86 \\
Depth 4 + Width 1:8 & WikiText & No & & 17.9 & 52.3 & 29.98 & 1.85 \\
Depth 4 + Width 1:8 & WikiText & Yes & & 14.5 & 45.6 & 32.79 & 0.86 \\
Depth 4 + Width 1:8 & TriviaQA & No & & 19.1 & 47.5 & 90.72 & 1.85 \\
Depth 4 + Width 1:8 & TriviaQA & Yes & & 17.1 & 38.3 & 99.25 & 0.86 \\
Depth 4 + Width 2:4 & WikiText & No & & 7.7 & 27.0 & 49.71 & 1.85 \\
Depth 4 + Width 2:4 & WikiText & Yes & & 7.2 & 25.8 & 52.20 & 0.86 \\
Depth 4 + Width 2:4 & WikiText & EoRA & & 7.6 & 26.3 & 51.19 & 0.86 \\
Depth 4 + Width 2:4 & TriviaQA & No & & 9.6 & 15.9 & 298.63 & 1.85 \\
Depth 4 + Width 2:4 & TriviaQA & Yes & & 8.5 & 16.4 & 319.14 & 0.86 \\
Depth 4 + Width 2:4 & TriviaQA & EoRA & & 8.6 & 17.3 & 308.11 & 0.86 \\
Depth 4 + Width 3:4 & WikiText & No & & 0.8 & 4.3 & 156.76 & 1.85 \\
Depth 4 + Width 3:4 & WikiText & Yes & & 1.1 & 4.6 & 164.28 & 0.86 \\
Depth 4 + Width 3:4 & TriviaQA & No & & 1.7 & 0.5 & 3652.35 & 1.85 \\
Depth 4 + Width 3:4 & TriviaQA & Yes & & 1.9 & 0.3 & 4545.42 & 0.86 \\
Depth 4 + Width 4:8 & WikiText & No & & 8.9 & 33.2 & 44.04 & 1.85 \\
Depth 4 + Width 4:8 & WikiText & Yes & & 8.2 & 29.6 & 46.34 & 0.86 \\
Depth 4 + Width 4:8 & WikiText & EoRA & & 8.3 & 31.1 & 45.53 & 0.86 \\
Depth 4 + Width 4:8 & TriviaQA & No & & 10.5 & 20.8 & 210.94 & 1.85 \\
Depth 4 + Width 4:8 & TriviaQA & Yes & & 11.1 & 19.1 & 226.30 & 0.86 \\
Depth 4 + Width 4:8 & TriviaQA & EoRA & & 11.6 & 18.0 & 219.34 & 0.86 \\
Depth 6 + Width 1:16 & WikiText & No & & 11.5 & 25.6 & 37.82 & 1.62 \\
Depth 6 + Width 1:16 & WikiText & Yes & & 9.6 & 22.8 & 40.73 & 0.80 \\
Depth 6 + Width 1:16 & WikiText & EoRA & & 9.4 & 24.8 & 39.63 & 0.80 \\
Depth 6 + Width 1:16 & TriviaQA & No & & 16.8 & 27.5 & 310.53 & 1.62 \\
Depth 6 + Width 1:16 & TriviaQA & Yes & & 14.6 & 25.8 & 269.79 & 0.80 \\
Depth 6 + Width 1:16 & TriviaQA & EoRA & & 15.8 & 25.4 & 292.86 & 0.80 \\
Depth 6 + Width 4:8 & WikiText & No & & 0.9 & 9.6 & 55.46 & 1.62 \\
Depth 6 + Width 4:8 & WikiText & Yes & & 2.1 & 10.7 & 58.80 & 0.80 \\
Depth 6 + Width 4:8 & TriviaQA & No & & 8.9 & 16.2 & 568.93 & 1.62 \\
Depth 6 + Width 4:8 & TriviaQA & Yes & & 7.5 & 14.0 & 515.99 & 0.80 \\
Depth 6 + Width 6:8 & WikiText & No & & 1.7 & 5.8 & 139.97 & 1.62 \\
Depth 6 + Width 6:8 & WikiText & Yes & & 1.3 & 3.9 & 153.72 & 0.80 \\
Depth 6 + Width 6:8 & TriviaQA & No & & 2.2 & 0.7 & 3273.94 & 1.62 \\
Depth 6 + Width 6:8 & TriviaQA & Yes & & 2.2 & 0.7 & 3457.97 & 0.80 \\
Depth 8 + Width 1:8 & WikiText & No & & 5.8 & 16.8 & 58.58 & 1.40 \\
Depth 8 + Width 1:8 & WikiText & Yes & & 5.3 & 14.9 & 63.33 & \textbf{0.74} \\
Depth 8 + Width 1:8 & TriviaQA & No & & 8.7 & 1.3 & 132836.59 & 1.40 \\
Depth 8 + Width 1:8 & TriviaQA & Yes & & 3.8 & 0.3 & 126753.56 & \textbf{0.74} \\
Depth 8 + Width 3:4 & WikiText & No & & 0.2 & 1.3 & 278.36 & 1.40 \\
Depth 8 + Width 3:4 & WikiText & Yes & & 0.5 & 1.3 & 289.45 & \textbf{0.74} \\
Depth 8 + Width 3:4 & TriviaQA & No & & 0.3 & 0.3 & 7263.56 & 1.40 \\
Depth 8 + Width 3:4 & TriviaQA & Yes & & 0.8 & 0.2 & 9326.59 & \textbf{0.74} \\


\end{longtable}
\normalsize

\subsection{Examples of Generated Text} \label{sec:generated_text}
To illustrate the qualitative differences between configurations, Table \ref{tab:generated_answers} contains several examples of text generated by some of these configurations. Each model has generated the text based on the prompt ``\texttt{Question: Who was the man behind The Chipmunks?\char`\\nAnswer:}".

The configurations are labeled using the initial letter of the name of the procedure (i.e. D[L] for depth pruning of L layers, W[N]:[M] for width pruning with N:M ratio, W and T for WikiText and TriviaQA LoRA fine-tuning respectively, Q for quantization and E for EoRA).

{\scriptsize
\begin{longtable}{@{}l p{\dimexpr\textwidth-3cm-4\tabcolsep}@{}}
\caption[Text Generation Examples]{Sample answers generated by selected model configurations.} \label{tab:generated_answers} \\
\hline
\textbf{Config} & \textbf{Generated Text} \\
\hline
\endfirsthead

\multicolumn{2}{c}%
{{\footnotesize \bfseries \tablename\ \thetable{} -- continued from previous page}} \\
\hline
\textbf{Config} & \textbf{Generated Text} \\
\hline
\endhead

\hline \multicolumn{2}{r}{{Continued on next page}} \\
\endfoot

\hline
\endlastfoot
Baseline Instruct & Ross Bagdasarian Sr. was an American singer, songwriter, and record producer who was known for his hit song "The Chipmunk Song [..] \\
W1:2 + T & John Paul (disambiguation) - The Chipmunk's Hat (disambiguation) - The Chipmunk's Hat (disambiguation) - The Chipmunk[..] \\
W1:2 + T + Q & The Chipmunk (disambiguation) is the English title of which film? \\
W1:2 + T + Q + E & John Paul (disambiguation) - The Chipmunks - The Chipmunks (disambiguation) - The Chipmunks - The Chipmunks - The Chi[..] \\
W1:2 + W & The Chipmunks are the only group of the world's most popular and influential rock bands. The group is known for its u[..] \\
W1:2 + W + Q & The Chipmunks are a group of children who are born in the early 20th century, who are the descendants of the family o[..] \\
W1:2 + W + Q + E & The Chipmunks are the only group of children who are known to be the only ones who have been born in the same way as [..] \\
W1:8 + T & Ross Bagdasarian Jr. (disambiguation) - film producer and singer. He was born Ross Bagdasarian, Jr. in 1936 and died [..] \\
W1:8 + W & Ross Bagdasarian Jr., also known as David Seville, is the voice behind the Chipmunks. He was born on May 24, 1943, in[..] \\
W1:8 + W + Q & The Chipmunk Man, also known as The Chipmunk Man, is a 2005 American comedy film directed by Chris Weigman and writte[..] \\
W4:8 + T & John Brown (disambiguation) (film) character. (disambiguation) John Brown (disambiguation) (film) character. (disambi[..] \\
W4:8 + T + Q & John Brown (disambiguation) - Wikipedia disambiguation page is for John Brown. He was the main character in the Chipm[..] \\
W4:8 + W & The Chipmunks were created by the British cartoonist and writer Arie Bellosi.  \\
W8:16 + T + Q & Dave Barry (disambiguation) and Neil Richardson (disambiguation) are both the founders of the company, but Dave Barry[..] \\
W8:16 + W + Q & The Chipmunks were created by the British animation studio Aardust Productions. The studio was founded in 1991 by Pet[..] \\
D2 + T + Q & Don Quibberta of the Chipmunks. Chipmunk song. Chipmunk song. [..] \\
D2 + W & The Chipmunks were created by Tom Pollock, a British animator and cartoonist. Pollock was inspired by the 1950s carto[..] \\
D2 + W1:16 + T + Q + E & Richard E. Young (disambiguation) - Wikipedia (disambiguation) - Wikipedia (disambiguation) - Wikipedia (disambiguati[..] \\
D2 + W1:16 + W & The Chipmunks were created by Tom Pollock, a British animator and cartoonist. Pollock was inspired by the 1950s carto[..] \\
D2 + W1:16 + W + Q & James Howard Morris, a British actor, was the main inspiration behind the character of Chip's father, who is the main[..] \\
D2 + W1:8 + W & The Chipmunks was created by Matt Williams, a former animator and voice actor for the Nickelodeon cartoon series The [..] \\
D2 + W1:8 + W + Q & James T. Green ( 1969 – 1980 )  \\
D2 + W1:8 + W + Q + E & The Chipmunks was created by the British cartoonist and animator, James T. Landry. He was inspired by the 1960s carto[..] \\
D2 + W2:4 + T + Q & The Chipmunks (disambiguation) (disambiguation) (reality test) (disambiguation) (reality test) (disambiguation) (disa[..] \\
D2 + W2:4 + W + Q & The Chipmunks were the " chipmunks ", a group of children who were the victims of the 1980s @-@ 80s. The Chipmunks we[..] \\
D2 + W3:4 + T & John James (disambiguation) (disambiguation) [..] \\
D2 + W3:4 + T + Q & The Chipcals (disambiguation) (disambiguation) [..] \\
D2 + W3:4 + T + Q + E & John James, John James, [..] \\
D2 + W3:4 + W & The Chipmunks's first answer, the first answer, [..] \\
D2 + W3:4 + W + Q & The Chipmunks's first two @-@ @-@ two [..] \\
D2 + W3:4 + W + Q + E & The Chipmunks's first answer, The Chipmunks's first answer, [..] \\
D2 + W4:8 + T & John C. Smith, Jr. (disambiguation) - Chipmunk Jr. (disambiguation) - Chipmunk Jr. (disambiguation) - Chipmunk Jr. (d[..] \\
D4 + W & The Chipmunks were created by Canadian comedian and musician John L. Linder, who was a member of the Canadian comedy [..] \\
D4 + W + Q & The Chipmunks were the men behind the Chipmunks cartoon series. Chipy, the Chipmunks'main character, was a chipster w[..] \\
D4 + W3:4 + T & John (disambiguation) (disambiguation) (disambiguation) (disambiguation) (disdisdisiguation) (disdisiguation) (disdis[..] \\
D4 + W3:4 + T + Q & The 1saurus (disambiguation) is the name of the world's highest point of the world? \\
D4 + W3:4 + W & The first question of the question, the question was the question of the question, and the question of the question w[..] \\
D4 + W3:4 + W + Q & The first question, the question was the question of the question, and the question was not not so much of the questi[..] \\
D4 + W4:8 + W & The Chipmunk was a young boy who was born in 1950. He was a young boy who was a member of the British Royal Navy, and[..] \\
D4 + W4:8 + W + Q & The Chipmunk was a 2000 film directed by the film director John Kostin. The film was directed by the film director Jo[..] \\
D6 + W1:16 + T + Q + E & Charles Dickens (disambiguation) and Dr. Jolly Jolly (disambiguation) and Dr. M. Snow (fiction) and Dr. Snow (fiction[..] \\
D6 + W1:16 + W & The Chipmunks, the original version of the cartoon, was created by the Canadian cartoonist John H. Griffith, who was [..] \\
D6 + W1:16 + W + Q & The Chipmunks, the original version of the cartoon, was created by the Canadian cartoonist Jeff Sarnick and the Ameri[..] \\
D6 + W3:4 + T & The Great Road (disambiguation) (disambiguation) (disambiguation)) (disambiguation)) (disambigig) and aresststlylylyl[..] \\
D6 + W3:4 + T + Q & The World World World World World [..] \\
D6 + W4:8 + T & John F. F. F. B. W. W. [..] \\
D6 + W6:8 + T & John O. (disambiguation) is the name of the 1990s film ‘The Great Man’? \\
D6 + W6:8 + T + Q & The Chip Chip, the Chip Chip, is the name for the first body of which body? \\
D6 + W6:8 + W & The Chipmunks of the 2000s were the first of the 2000s in the 2000s. The 2000s were [..] \\
D6 + W6:8 + W + Q & The Chipmunks of the 2000s were the first of the 2000s in the 2000s. The Chipimunks of the 2000s [..] \\
D8 + W & The Chipmunks, the first child of the family, was the first child of the [..] \\
D8 + W + Q & The Chipmunks were the first to record the first recording of the song " The Chipmunks ". The song was recorded by th[..] \\
D8 + W + Q + E & The Chipmunks were the first to be recorded in the 1970s. The Chipmunks were the first to be recorded in the 1970s. T[..] \\
D8 + W1:8 + W & The Chipmunks were the first to be recorded in the 1980s by the BBC. The Chipmunks were the first to be recorded in t[..] \\
D8 + W3:4 + T & John (disambiguation) and the film of the film The The[..] \\
D8 + W3:4 + T + Q & The Black Black Black Black Black [..] \\
D8 + W3:4 + W & the first first @-@ 5 @-@ 5 @-@ 5 [..] \\
D8 + W3:4 + W + Q & the 11ththththththththththththththththth[..] \\

\end{longtable}
}
\normalsize

\chapter{Alternative LoRA Integration Strategies}
\label{app:appendix2}
This appendix examines the effects of applying LoRA adaptation earlier in the compression pipeline, specifically after depth pruning but before width pruning. The modified sequence decouples depth and width pruning, while mitigating the interference between LoRA weight modifications and WANDA's importance-based pruning decisions.

\subsection{Experimental Results}
[explain here time constraints and how they did not allow for extensive experiments]

\begin{center}
\captionof{table}{Performance Comparison: Standard vs. Early LoRA Integration}
\label{tab:lora_positioning}
\small
\begin{tabular}{llcccc}
\hline
\multirow{2}{*}{\textbf{Config}} & \multirow{2}{*}{\textbf{LoRA}} & \multicolumn{2}{c}{\textbf{TriviaQA (\%)}} & & \textbf{WikiText-2} \\
\cline{3-4}
& & \textbf{Closed} & \textbf{Open} & & \textbf{Perplexity} \\
\hline
LLaMA-3.2-1B-Instruct & None & XX.X & XX.X & & XX.XX \\
Standard & Post-Prun. & XX.X & XX.X & & XX.XX \\
Early & Pre-Width & XX.X & XX.X & & XX.XX \\
\hline
\end{tabular}
\end{center}


\chapter*{Acknowledgements}
\markboth{ACKNOWLEDGEMENTS}{ACKNOWLEDGEMENTS}

After this long journey, I can't believe I'm finally closing this chapter of my life. The past few years have been filled with big accomplishments and just as big disappointments, which have definitely made me rethink my priorities and grow as a person. On the other hand, the last 10 months have made all the difference: a fire has reignited in my soul and I've created some of the best projects I've ever made, work I'm really proud of. Now I have this insatiable hunger for greatness, and I can't wait to show the world what I'm capable of when I put my mind to it.

Now, shifting focus to the people who made this possible. The following section is for those I want to thank and who I'm incredibly grateful to know. Buckle up, because this is gonna be a long one.

I want to thank Professor Conti, who supervised this thesis and shared invaluable insights on model optimization best practices. We're perfectly aligned in our vision, since model optimization is a topic I hold very dear to my heart.

Next, I want to thank Luca Bompani, an amazing co-supervisor. He helped me a lot with both the thesis and the project, while also welcoming my ideas and providing valuable criticism when I was at my peak Dunning-Kruger, which I greatly appreciated.

Obviously, I thank my parents, who gave me unconditional support along the way, and my brother. This thesis is dedicated to them.

A special thank you goes to my friend Leon. He is probably one of the best people anyone could ever meet, and he had my back whenever I was at my lowest. I genuinely wish for everyone to know someone as kind, gentle and \textit{so chill} as him. Of course I also thank Tony, Giaco, Donnoh, Fred, Massaro and Kevin: they're all really great guys with remarkable energy who led me to some very strange adventures (in a good way!).

I am grateful to Galf, my friend since high school with whom I've shared many beautiful moments, and Profex, who is literally the most talented chemist in Italy.

A big thank you goes to my calisthenics group, ``Bimbe di Ruggio''. Those guys are genuinely amazing people who gave me a ton of motivation and have occupied a reasonable chunk of the last 3 years, while keeping me fit at the same time!

Now this paragraph is dedicated to the 2 months I spent at CERN in the summer of 2023, which have been the best months of my life so far. I want to thank all the ``Choccy Guys'' and ``Cool Guys'' from CERN. I thank Guilherme, my supervisor there who was not just a great mentor, but a very good friend as well. A huge thank you to Alina, Aya, Bruno, Daniel, Eliacim, Gadea, Guilherme (not the supervisor, this is a different one), Henar, Inés, Jessica, Joaquin, Jordi, Luis, Matìas, Matthias, Melike, Niko, Omar, Pablo, Rim, Stefan, Thomas, and many others. It's incredible how some people can change your life so much just by knowing them for such a short time.

Last, but definitely not least, I want to thank Letizia, who is the greatest travel companion anyone could have, and who is probably the most kind-hearted person in the world. She's incredibly caring and has been there for me through everything. She's also the de facto third supervisor of this thesis. Thank you very much Leti.

Wow, that \textit{really} was quite long, and there are so many more I could mention. But sitting here, I realize something: all those disappointments, all those moments I wanted to quit, they all led me to this moment, and these incredible people made every struggle worth it. And this? This is just the beginning.

Let's see what this ``greatness'' is all about.
\end{document}
